%A classe é a dcc-nce, e o parâmetro a ser informado é diss (para dissertação de mestrado)
\documentclass[diss]{dcc-nce}
\usepackage[T1]{fontenc}
\usepackage{color,graphicx}
\usepackage{graphics}
% Caminho para as imagens que serão inseridas no documento
\graphicspath{{./imagens/}}

\usepackage{url}
\usepackage{amsmath,amssymb}
%\usepackage[numbers]{natbib}
\usepackage{natbib}
\usepackage{dsfont} %Usado para conjuntos N, Z, Q, R, C

\usepackage[portuguese,algoruled,longend]{algorithm2e}
% Corrige problemas de tradução para português do pacote algorithm2e
% Vide http://tex.stackexchange.com/questions/113325/problem-with-algorithm2e-and-portuguese-option
\SetKwFor{Para}{Para}{fa\c{c}a}{fim para}
\SetKwIF{Se}{SenaoSe}{Senao}{Se}{então}{senão, se}{senão}{fim}
\SetKwBlock{Iniciacao}{Inicia\c{c}ão}{}{}
\SetKw{Return}{retorne}
\SetKwFor{Enquanto}{Enquanto}{faça}{fim enquanto}

\usepackage{algorithmic}
\usepackage[utf8]{inputenc}
\usepackage{listings}%Para inserir codigos fontes de programas no apendice.
\usepackage{xcolor}
% Definindo novas cores
\definecolor{verde}{rgb}{0,0.5,0}
% Configurando layout para mostrar codigos C++
\usepackage{listings}
\lstset{
  language=C++,
  basicstyle=\ttfamily\small,
  keywordstyle=\color{blue},
  stringstyle=\color{verde},
  commentstyle=\color{red},
  extendedchars=true,
  showspaces=false,
  showstringspaces=false,
  numbers=left,
  numberstyle=\tiny,
  breaklines=true,
  backgroundcolor=\color{green!10},
  breakautoindent=true,
  captionpos=b,
  xleftmargin=0pt,
}

% Para contagem do numero total de folhas:
\usepackage{everyshi}
\makeatletter
\let\totalpages\relax
\newcounter{mypage}
\EveryShipout{\stepcounter{mypage}}
\AtEndDocument{\clearpage
   \immediate\write\@auxout{%
    \string\gdef\string\totalpages{\themypage}}}
\makeatother


\topmargin=0in
\textheight=20.5cm


\begin{document}

%Este tem que vir primeiro neste arquivo, caso contrario nao aparecerao
%as palavras-chave na ficha catalografica:
\input{pre-textual/palavrasChavePortugues}    % Editar o arquivo palavrasChavePortugues

%O restante vem depois:
\title{Coloque aqui o Título da sua dissertação}

% COLOCAR AQUI O CÓDIGO FORNECIDO PELA BIBLIOTECA DO NCE PARA A FICHA CATALOGRÁFICA:
\codigobiblioteca{CÓDIGO DA BIBLIOTECA}

\author{Último Sobrenome do Aluno}{Nome Sobrenome1 Sobrenome2 ...}
\advisor[Prof.~Dr.~]{Último Sobrenome}{Nome Sobrenome1 Sobrenome2 ...}
\coadvisor[Profa. Dra.]{Último Sobrenome}{Nome Sobrenome1 Sobrenome2 ...}
\banca{Profa. Dr. Fulano de Tal}{Prof. Dr. Beltrano}{Prof. Dr. Ciclano}
\date{2013}
\maketitle
                      % Editar o arquivo capa.tex
\input{pre-textual/dedicatoria}               % Editar o arquivo dedicatoria.tex
\input{pre-textual/agradecimentos}            % Editar o arquivo agradecimentos.tex
\begin{abstract}
Resumo em português.
\end{abstract}
           % Editar o arquivo resumoPortugues.tex
%\begin{englishabstract}{Colocar aqui o título da dissertação em inglês}{coloque aqui as palavras-chave em inglês, separadas por vírgula}

\begin{englishabstract}{}{coloque aqui as palavras-chave em inglês, separadas por vírgula}

Colocar aqui o resumo em inglês.

\end{englishabstract}

              % Editar o arquivo resumoIngles.tex

\listoffigures{}

\listoftables{}

%\input{pre-textual/listaAbreviaturaSiglas}    % Editar o arquivo listaAbreviaturaSiglas.tex

\tableofcontents{}                  % Sumário

\parindent=1.25cm %Início de cada parágrafo a partir da margem esquerda
\parskip=20pt
\baselineskip=20pt

\input{capitulos/capituloA.tex}  %Editar o capituloA
\input{capitulos/capituloB.tex}  %Editar o capituloB
\input{capitulos/conclusao.tex}  %Editar a conclusao
\input{capitulos/introducao.tex} %Editar a introducao

\bibliography{pos-textual/referencias}          % Editar o arquivo "referencias.bib"
\bibliographystyle{latex-stuff/abnt-ufrgs}      % Procura pelo arquivo "abnt-ufrgs" - normas ABNT.

\clearpage

\appendix
\input{pos-textual/apendice}          % Editar o arquivo apendice.tex

\end{document}

%\citep* - citação completa, com todos os autores

%\citeyearpar - cita somente o ano da publicação
