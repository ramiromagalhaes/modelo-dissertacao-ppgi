%A classe é a dcc-nce, e o parâmetro a ser informado é diss (para dissertação de mestrado)
\documentclass[diss]{dcc-nce}

\usepackage[T1]{fontenc}
\usepackage[utf8]{inputenc}

\usepackage{cmap} % Mapear caracteres especiais no PDF

\usepackage{graphicx}
\graphicspath{{./figuras/}} % Caminho padrão das imagens

\usepackage{xcolor}

% \usepackage[numbers]{natbib}
\usepackage{natbib}

\usepackage{amsmath,amssymb,amsfonts,textcomp}
\usepackage[portuguese,algoruled,longend]{algorithm2e}
\SetKwFor{Para}{Para}{fa\c{c}a}{fim para}                      % Corrige problemas de tradução para português do pacote algorithm2e
\SetKwIF{Se}{SenaoSe}{Senao}{Se}{então}{senão, se}{senão}{fim} % Vide http://tex.stackexchange.com/questions/113325/problem-with-algorithm2e-and-portuguese-option
\SetKwBlock{Iniciacao}{Inicia\c{c}ão}{}{}
\SetKw{Return}{retorne}
\SetKwFor{Enquanto}{Enquanto}{faça}{fim enquanto}

\usepackage{url}

% Suporte a lista de afazeres. Bem conveniente para anotar e exibir pendências.
\usepackage{todonotes}

% Para contagem do numero total de folhas:
\usepackage{everyshi}
\makeatletter
\let\totalpages\relax
\newcounter{mypage}
\EveryShipout{\stepcounter{mypage}}
\AtEndDocument{\clearpage
   \immediate\write\@auxout{%
    \string\gdef\string\totalpages{\themypage}}}
\makeatother


% Suporte para hipertexto, links para referências e figuras.
\usepackage{hyperref} %Comentado pois ainda não está configurado corretamente...
% Configurações dos links e metatags do PDF a ser gerado
\hypersetup{colorlinks=true, linkcolor=blue, citecolor=blue, filecolor=blue, urlcolor=blue}


%Inclui as definições de macros e qualquer outra coisa importante para o usuário
\include{definicoes}


%Para informações sobre o layout da página, vide http://en.wikibooks.org/wiki/LaTeX/Page_Layout
\setlength{\topmargin}{0in}
\setlength{\textheight}{20.5cm}



\begin{document}

%Este tem que vir primeiro neste arquivo, caso contrario nao aparecerao
%as palavras-chave na ficha catalografica:
\input{pre-textual/palavrasChavePortugues}    % Editar o arquivo palavrasChavePortugues

%O restante vem depois:
\title{Coloque aqui o Título da sua dissertação}

% COLOCAR AQUI O CÓDIGO FORNECIDO PELA BIBLIOTECA DO NCE PARA A FICHA CATALOGRÁFICA:
\codigobiblioteca{CÓDIGO DA BIBLIOTECA}

\author{Último Sobrenome do Aluno}{Nome Sobrenome1 Sobrenome2 ...}
\advisor[Prof.~Dr.~]{Último Sobrenome}{Nome Sobrenome1 Sobrenome2 ...}
\coadvisor[Profa. Dra.]{Último Sobrenome}{Nome Sobrenome1 Sobrenome2 ...}
\banca{Profa. Dr. Fulano de Tal}{Prof. Dr. Beltrano}{Prof. Dr. Ciclano}
\date{2013}
\maketitle
                      % Editar o arquivo capa.tex
\input{pre-textual/dedicatoria}               % Editar o arquivo dedicatoria.tex
\input{pre-textual/agradecimentos}            % Editar o arquivo agradecimentos.tex
\begin{abstract}
Resumo em português.
\end{abstract}
           % Editar o arquivo resumoPortugues.tex
%\begin{englishabstract}{Colocar aqui o título da dissertação em inglês}{coloque aqui as palavras-chave em inglês, separadas por vírgula}

\begin{englishabstract}{}{coloque aqui as palavras-chave em inglês, separadas por vírgula}

Colocar aqui o resumo em inglês.

\end{englishabstract}

              % Editar o arquivo resumoIngles.tex

\listoffigures{}

\listoftables{}

%\input{pre-textual/listaAbreviaturaSiglas}    % Editar o arquivo listaAbreviaturaSiglas.tex

\tableofcontents{}                             % Sumário

\setlength{\parindent}{1.25cm}                 % Início de cada parágrafo a partir da margem esquerda
\setlength{\parindent}{20pt}
\setlength{\baselineskip}{20pt}

\pagenumbering{arabic} %numeração de páginas em arábico começa a partir do primeiro capítulo

\input{capitulos/introducao.tex} %Editar a introducao
\input{capitulos/capituloA.tex}  %Editar o capituloA
\input{capitulos/capituloB.tex}  %Editar o capituloB
\input{capitulos/conclusao.tex}  %Editar a conclusao

\bibliography{pos-textual/referencias}    % Arquivo .BIB com as referências bibliográficas
\bibliographystyle{latex-stuff/abnt-ufrgs} % Arquivo de descreve como as referências devem ser apresentadas

\clearpage

%\appendix %comentado pois não está funcionando corretamente...
%\input{pos-textual/apendice}          % Editar o arquivo apendice.tex

\end{document}

