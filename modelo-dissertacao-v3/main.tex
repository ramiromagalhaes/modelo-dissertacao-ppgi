%A classe é a dcc-nce, e o parâmetro a ser informado é diss (para dissertação de mestrado)
\documentclass[diss]{dcc-nce}
\usepackage[utf8]{inputenc}
\usepackage[T1]{fontenc}
\usepackage{color,graphicx}
\usepackage{graphics}
\usepackage{url}
\usepackage{amsmath,amssymb}
\usepackage[numbers]{natbib}

\topmargin=0in
\textheight=20.5cm

\begin{document}

\input{pre-text/palavrasChavePortugues}    % Editar o arquivo palavrasChavePortugues
\title{Coloque aqui o Título da sua dissertação}

% COLOCAR AQUI O CÓDIGO FORNECIDO PELA BIBLIOTECA DO NCE PARA A FICHA CATALOGRÁFICA:
\codigobiblioteca{CÓDIGO DA BIBLIOTECA}

\author{Último Sobrenome do Aluno}{Nome Sobrenome1 Sobrenome2 ...}
\advisor[Prof.~Dr.~]{Último Sobrenome}{Nome Sobrenome1 Sobrenome2 ...}
\coadvisor[Profa. Dra.]{Último Sobrenome}{Nome Sobrenome1 Sobrenome2 ...}
\banca{Profa. Dr. Fulano de Tal}{Prof. Dr. Beltrano}{Prof. Dr. Ciclano}
\date{2013}
\maketitle
                      % Editar o arquivo capa.tex
\input{pre-text/dedicatoria}               % Editar o arquivo dedicatoria.tex
\input{pre-text/agradecimentos}            % Editar o arquivo agradecimentos.tex
\begin{abstract}
Resumo em português.
\end{abstract}
           % Editar o arquivo resumoPortugues.tex
\begin{englishabstract}{}{coloque aqui as palavras-chave em inglês, separadas por vírgula}

Colocar aqui o resumo em inglês.
\end{englishabstract}

              % Editar o arquivo resumoIngles.tex

\listoffigures{}

\listoftables{}

\input{pre-text/listaAbreviaturaSiglas}    % Editar o arquivo listaAbreviaturaSiglas.tex

\tableofcontents{}                  % Sumário

\parindent=1.25cm                   % Left margin for each paragraph
\parskip=20pt
\baselineskip=20pt

\input{chapters/introducao}         % Editar o arquivo introducao.tex
\input{chapters/capituloA}          % Editar o arquivo capituloA.tex
\input{chapters/capituloB}          % Editar o arquivo capituloB.tex
\input{chapters/conclusao}          % Editar o arquivo introducao.tex

\bibliography{referencias}          % Editar o arquivo "referencias.bib"
\bibliographystyle{abnt-ufrgs}      % Procura pelo arquivo "abnt-ufrgs" - normas ABNT.

\appendix
\input{post-text/apendice}          % Editar o arquivo apendice.tex

\end{document}

%\citep* - citação completa, com todos os autores

%\citeyearpar - cita somente o ano da publicação
